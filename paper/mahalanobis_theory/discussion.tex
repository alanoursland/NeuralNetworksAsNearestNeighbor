% discussion.tex

\section{Implications and Discussion}
\label{sec:discussion}

We discuss implications, potential impact and future work of this reframing of linear layers in neural networks. While this paper provides a robust theoretical foundation for interpreting neural networks through Mahalanobis distance and Abs activation functions, it does not include empirical results. Future work will involve validating these theoretical insights with empirical data to further assess their applicability and performance in real-world scenarios.

\subsection{Expected Value Interpretation}

The expected value, or mean, is a central concept in statistics, representing the average tendency of a distribution. In neural networks, finding the expected value for each neuron would reveals the features it recognizes. Interpreting linear nodes as approximations of Gaussian principal components  provides a path towards recoving the neuron mean value. The estimated mean serves as a prototype for the feature that the neuron has learned to recognize \citep{li2018deep}, representing the 'ideal' input for that neuron. This interpretation enhances the transparency of the feature extraction process, potentially leading to more interpretable models and improved architectures.

\subsection{Equivalence between Abs and ReLU Activations}

While our analysis utilizes linear layers with Abs activation functions to model deviations along principal component directions, ReLU activations can provide comparable information within the same framework.

For the \emph{Abs activation}, each linear node computes:
\begin{equation}
\label{eq:abs_activation_compact}
y_{\text{Abs}} = \left| \mathbf{w}^\top \mathbf{x} + b \right|,
\end{equation}
where the weights \(\mathbf{w}\) and bias \(b\) are set such that \(\mathbf{w}^\top \boldsymbol{\mu} + b = 0\). This centers the decision boundary at the cluster mean \(\boldsymbol{\mu}\), and within a confidence interval \(\delta\), the pre-activation output ranges from \(-\delta\) to \(+\delta\).

For the \emph{ReLU activation}, we adjust the bias to shift the decision boundary just outside the cluster:
\begin{equation}
\label{eq:relu_activation_compact}
y_{\text{ReLU}} = \max\left( 0, -\mathbf{w}^\top \mathbf{x} - b + \delta \right).
\end{equation}
Here, the pre-activation output ranges from \(0\) to \(2\delta\) within the cluster. Although ReLU zeros out negative inputs, by negating the pre-activation and adjusting the bias, it effectively captures the magnitude of deviations similar to the Abs activation.

The hyperplanes defined by \(\mathbf{w}\) maintain the same orientation in both cases, providing equivalent views of the cluster. Subsequent layers can adapt to either activation's output range, making Abs and ReLU functionally comparable in capturing essential features.

This suggests that techniques developed for networks with Abs activations may be adaptable to ReLU activations, bridging theoretical insights with practical neural architectures commonly utilizing ReLU.

\subsection{Activations as Distance Metrics}

Traditional neural networks typically employ an ``intensity metric model,'' where larger activation values indicate stronger feature presence. In contrast, a ``distance metric model'' interprets smaller activation values as indicating closer proximity to a learned feature or prototype. The following observations suggest directions for future work:

\begin{itemize}
    \item Most error functions (e.g., Cross Entropy Loss, Hinge Loss) are designed for intensity metrics. Output layers using Abs activation may require modification of their output values.
    \item While some architectures, like Radial Basis Function networks \citep{broomhead1988radial}, utilize distance metrics, they are not widely adopted in modern deep learning.
    \item Distance metrics conflict with the goal of sparse output layers. In a distance metric model, zero is the strongest signal, making it illogical for most outputs to have the strongest signal.
    \item The Gaussian connection suggests transforming distance metrics through exponential ($y=e^{-x^2}$) or Laplace ($y=e^{-|x|}$) functions to convert them into intensity metrics. However, these may suffer from vanishing gradients. A approximation of these functions could combine Abs and ReLU: $y=\text{ReLU}(-\text{Abs}(x) + \text{confidence\_bound})$.
    \item Distance and intensity metrics can be interconverted through negation. Subsequent layer weights can apply their own negation, obscuring the metric type learned by internal nodes.
    \item There may exist regularization techniques that encourage distance metric learning \citep{weinberger2009distance}.
\end{itemize}

\subsection{Model Initialization and Pretraining}

Interpreting neurons as learning distances from cluster means suggests novel approaches to model initialization and pretraining. This perspective offers an alternative to standard random initialization techniques \citep{kamilov2017survey} by incorporating data-driven insights into the model's starting configuration.

Rather than initializing with random weights, an approach could involve clustering the input data (e.g., using k-means) and calculating the covariance of each cluster. Applying Principal Component Analysis (PCA) to these covariance matrices can provide a basis for directly initializing network parameters. This strategy leverages the structure of the data to guide the network's early learning stages. This process, and approximations of this process, may offer several advantages:

\begin{itemize} \item Faster convergence by starting with parameters informed by the data distribution \item Enhanced interpretability, as network weights are aligned with meaningful features from the outset \item Improved generalization by incorporating information about cluster structures \end{itemize}

\subsection{Model Translation and Componentization}

The interpretation of neurons as principal components of Gaussians suggests a potential mapping between neural networks and hierarchical Gaussian Mixture Models (GMMs) \citep{jacobs1991adaptive}. By performing PCA on the clusters in a GMM, we can extract principal components and means, converting them directly into neurons. This possibility of directly translating neural networks into GMMs offers several potential advantages:

\begin{itemize}
    \item \textbf{Enhanced Interpretability}: Neural networks can be better understood through their GMM equivalents, providing insights into the data distribution and feature representations.
    \item \textbf{Application of Statistical Techniques}: Established statistical methods used in GMM analysis can be applied to neural networks, potentially improving training and evaluation.
    \item \textbf{Hybrid Models}: Combining neural networks and GMMs can leverage the strengths of both, enhancing performance in tasks like clustering and classification.
    \item \textbf{Model Decomposition}: Large networks might be decomposable into smaller, context-specific subnetworks, facilitating easier analysis and maintenance.
    \item \textbf{Efficient Storage and Computation}: Subnetworks can be stored offline and dynamically loaded based on data context, improving memory efficiency and reducing computational overhead.
    \item \textbf{Scalability in Large-Scale Applications}: This approach can lead to faster inference and more efficient resource utilization in applications dealing with massive datasets.
\end{itemize}

\subsection{Direct use of Mahalanobis equation}

Equation \ref{eq:mahalanobis_single_component} explicitly incorporates the variance eigenvalue $\lambda$, the unit eigenvector $\mathbf{v}$, and the mean $\boldsymbol{\mu}$. Batch Normalization already makes use of $\lambda$ and $\boldsymbol{\mu}$ \citep{ioffe2015batch}, while the nGPT model employs unit weight vectors, which are analogous to $\mathbf{v}$ \citep{loshchilov2024ngptnormalizedtransformerrepresentation}. The success of these techniques suggest there might be further opportunities to decompose the standard linear layer equation $y = Wx + b$ towards the Mahalanobis equation in a way that leads to improvements in training speed and representation quality.

