% discussion.tex

\section{Implications and Discussion}
\label{sec:discussion}

We discuss implications, potential impact and future work of this reframing of linear layers in neural networks. While this paper provides a robust theoretical foundation for interpreting neural networks through Mahalanobis distance and Abs activation functions, it does not include empirical results. Future work will involve validating these theoretical insights with empirical data to further assess their applicability and performance in real-world scenarios.

\subsection{Expected Value Interpretation}

The expected value, or mean, is a central concept in statistics, representing the average tendency of a distribution. In neural networks, finding the expected value for each neuron would reveals the features it recognizes. Interpreting linear nodes as approximations of Gaussian principal components  provides a path towards recoving the neuron mean value. The estimated mean serves as a prototype for the feature that the neuron has learned to recognize \citep{li2018deep}, representing the 'ideal' input for that neuron. This interpretation enhances the transparency of the feature extraction process, potentially leading to more interpretable models and improved architectures.

\subsection{Equivalence between Abs and ReLU}

While our analysis of linear layers suggests the use of Abs activation, we propose that ReLU can provide comparable information and may be interpretable within the same framework. Given a confidence bound $\delta$, we observe:

\begin{itemize}
    \item For Abs activation: The preceding linear layer learns to output $\{-\delta, +\delta\}$, with the decision boundary intersecting the cluster mean.
    \item For ReLU activation: The preceding linear layer can position its decision boundary just outside the cluster, learning to output values between $\{0, 2\delta\}$.
\end{itemize}

Subsequent layers in the network can adapt to either output range. This suggests a functional equivalence between Abs and ReLU in this context. Techniques developed to enhance learning and interpretation with Abs activation functions may be adaptable to ReLU, potentially bridging theoretical insights with practical neural architectures.

\subsection{Activations as Distance Metrics}

Traditional neural networks typically employ an ``intensity metric model,'' where larger activation values indicate stronger feature presence. In contrast, a ``distance metric model'' interprets smaller activation values as indicating closer proximity to a learned feature or prototype. The following observations suggest directions for future work:

\begin{itemize}
    \item Most error functions (e.g., Cross Entropy Loss, Hinge Loss) are designed for intensity metrics. Output layers using Abs activation may require modification of their output values.
    \item While some architectures, like Radial Basis Function networks \citep{broomhead1988radial}, utilize distance metrics, they are not widely adopted in modern deep learning.
    \item Distance metrics conflict with the goal of sparse output layers. In a distance metric model, zero is the strongest signal, making it illogical for most outputs to have the strongest signal.
    \item The Gaussian connection suggests transforming distance metrics through exponential ($y=e^{-x^2}$) or Laplace ($y=e^{-|x|}$) functions to convert them into intensity metrics. However, these may suffer from vanishing gradients. A approximation of these functions could combine Abs and ReLU: $y=\text{ReLU}(-\text{Abs}(x) + \text{confidence\_bound})$.
    \item Distance and intensity metrics can be interconverted through negation. Subsequent layer weights can apply their own negation, obscuring the metric type learned by internal nodes.
    \item There may exist regularization techniques that encourage distance metric learning \citep{weinberger2009distance}.
\end{itemize}

\subsection{Model Initialization and Pretraining}

Interpreting neurons as learning distances from cluster means suggests novel approaches to model initialization and pretraining. This perspective offers an alternative to standard random initialization techniques \citep{kamilov2017survey} by incorporating data-driven insights into the model's starting configuration.

Rather than initializing with random weights, an approach could involve clustering the input data (e.g., using k-means) and calculating the covariance of each cluster. Applying Principal Component Analysis (PCA) to these covariance matrices can provide a basis for directly initializing network parameters. This strategy leverages the structure of the data to guide the network's early learning stages. This process, and approximations of this process, may offer several advantages:

\begin{itemize} \item Faster convergence by starting with parameters informed by the data distribution \item Enhanced interpretability, as network weights are aligned with meaningful features from the outset \item Improved generalization by incorporating information about cluster structures \end{itemize}

\subsection{Model Translation and Componentization}

The interpretation of neurons as principal components of Gaussians suggests a potential homomorphism between neural networks and hierarchical Gaussian Mixture Models (GMMs) \citep{jacobs1991adaptive}. This opens the possibility of directly translating neural networks into GMMs, offering several potential advantages:

\begin{itemize}
    \item Enhanced interpretability of neural networks through their GMM equivalents. 
    \item Ability to apply established statistical techniques from GMM analysis to neural networks.
    \item Potential for hybrid models that combine the strengths of both neural networks and GMMs.
    \item Decomposition of large networks into smaller, context-specific subnetworks.
    \item Offline storage of these subnetworks, with dynamic loading based on data context.
    \item Potential for improved memory efficiency and faster inference in large-scale applications.
\end{itemize}

\subsection{Direct use of Mahalanobis equation}

Equation \ref{eq:mahalanobis_single_component} explicitly incorporates the variance eigenvalue $\lambda$, the unit eigenvector $\mathbf{v}$, and the mean $\boldsymbol{\mu}$. Batch Normalization already makes use of $\lambda$ and $\boldsymbol{\mu}$ \citep{ioffe2015batch}, while the nGPT model employs unit weight vectors, which are analogous to $\mathbf{v}$ \citep{loshchilov2024ngptnormalizedtransformerrepresentation}. Decomposing the standard linear layer equation $y = Wx + b$ in a manner similar to the Mahalanobis equation may lead to improvements in both training speed and representation quality.

