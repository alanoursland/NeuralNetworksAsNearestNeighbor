% discussion.tex

\section{Implications and Discussion}
\label{sec:discussion}

We discuss implications, potential impact and future work of this reframing of linear layers in neural networks. While this paper provides a robust theoretical foundation for interpreting neural networks through Mahalanobis distance and Abs activation functions, it does not include empirical results. Future work will involve validating these theoretical insights with empirical data to further assess their applicability and performance in real-world scenarios.

\subsection{Expected Value Interpretation}

The concept of expected value or mean is fundamental in statistics and machine learning, providing a central tendency that characterizes a distribution. In the context of neural networks, identifying an expected value for each neuron can offer insights into the features it recognizes and how it processes information.

By interpreting linear nodes as principal components of Gaussian approximations of data clusters, we can potentially identify a mean value for the data points recognized by each neuron. This interpretation provides a probabilistic perspective on neural network operations and enhances our understanding of their internal representations.

As a linear separator, a neuron with inputs in $\mathbb{R}^n$ defines a hyperplane with $(n-1)$ dimensions and a normal vector in 1 dimension. The normal vector represents the Gaussian direction, with the mean being somewhere on the hyperplane. There are many techniques that could be successful at this including projecting data onto the normal vector and clustering, weighted averages using functions of the inverse distances, boundary constraints, optimization searches and manifold analysis.

This expected value or mean effectively acts as a prototype for the feature that the neuron has learned to recognize \citep{li2018deep}. It represents the 'ideal' or 'typical' input for that neuron, around which the neuron's response varies. This prototype interpretation can provide valuable insights into the feature extraction process of neural networks and may lead to more interpretable models and improved network architectures.

\subsection{Equivalence between Abs and ReLU}

This work stems from a desire to interpret neural network internals. While our analysis of standard MLP architecture leads to an L1 approximation of the Mahalanobis distance, necessitating an Abs activation, we posit that ReLU can provide comparable information and may be interpretable within the same framework.

Given a confidence bound $\delta$, we observe:

\begin{itemize}
    \item For Abs activation: The preceding linear layer learns to output $\{-\delta, +\delta\}$, with the decision boundary intersecting the cluster mean.
    \item For ReLU activation: The preceding linear layer can position its decision boundary just outside the cluster, learning to output $\{0, 2\delta\}$.
\end{itemize}

Subsequent layers in the network can adapt to either output range. This suggests a functional equivalence between Abs and ReLU in this context.

Techniques developed to enhance learning and interpretation with Abs activation functions may be adaptable to the more commonly used ReLU, potentially bridging theoretical insights with practical neural architectures.

\subsection{Activations as Distance Metrics}

Traditional neural networks typically employ an ``intensity metric model,'' where larger activation values indicate stronger feature presence. In contrast, a ``distance metric model'' interprets smaller activation values as indicating closer proximity to a learned feature or prototype \citep{broomhead1988radial}. The following observations suggest directions for future work:

\begin{itemize}
    \item Distance and intensity metrics can be interconverted through negation.
    \item Subsequent layer weights can apply their own negation, obscuring the metric type learned by internal nodes.
    \item Distance metrics are incompatible with sparse layer output.
    \item Most error functions (e.g., Cross Entropy Loss, Hinge Loss) are designed for intensity metrics. Output layers using Abs activation should be modified accordingly.
    \item The Gaussian connection suggests transforming distance metrics through exponential ($y=e^{-x^2}$) or Laplace ($y=e^{-|x|}$) functions. These may suffer from vanishing gradients. An alternative approximation could combine Abs and ReLU: $y=\text{ReLU}(-\text{Abs}(x) + \text{confidence\_bound})$.
    \item Some neural network architectures, such as Radial Basis Function networks \citep{broomhead1988radial}, have employed distance metrics, but they are not widely adopted.
    \item There may exist regularization techniques that encourage distance metric learning \citep{weinberger2009distance}.
\end{itemize}

\subsection{Model Initialization and Pretraining}

The interpretation of neurons as learning distances from cluster means suggests novel approaches to model initialization and pretraining. This perspective offers a potential alternative to standard random initialization techniques \citep{kamilov2017survey}.

Given randomly initialized weights $\mathbf{W}$, we propose setting the bias $b$ such that the decision boundary passes through a data point or cluster centroid:

\begin{equation}
    b = -\mathbf{W} \cdot \boldsymbol{\mu}
\end{equation}

where $\boldsymbol{\mu}$ represents a chosen data point or cluster centroid.

This initialization strategy could offer several potential advantages:

\begin{itemize}
    \item Faster convergence by starting with meaningful decision boundaries
    \item Improved interpretability of initial network states
    \item Potential for better generalization by incorporating data distribution information from the start
\end{itemize}

Future work could explore:
\begin{itemize}
    \item Empirical comparisons with standard initialization techniques (e.g., He, Xavier)
    \item Extensions to deep networks, considering layerwise or global clustering approaches
    \item Combination with other pretraining methods, such as autoencoders or contrastive learning
    \item Theoretical analysis of the impact on gradient flow and optimization dynamics
\end{itemize}

\subsection{Model Translation and Componentization}

The interpretation of neurons as principal components of Gaussians suggests a potential homomorphism between neural networks and hierarchical Gaussian Mixture Models (GMMs) \citep{jacobs1991adaptive}. This perspective opens up several promising avenues for research and application.

It may be possible to directly convert between neural networks and GMMs. This translation could offer several benefits:

\begin{itemize}
    \item Enhanced interpretability of neural networks through their GMM counterparts
    \item Ability to leverage well-established statistical techniques for GMMs in neural network analysis
    \item Potential for hybrid models that combine the strengths of both paradigms
\end{itemize}

The locality properties of Gaussians suggest a novel approach to managing large neural networks:

\begin{itemize}
    \item Decomposition of large networks into smaller, context-specific subnetworks
    \item Offline storage of these subnetworks, with dynamic loading based on data context
    \item Potential for improved memory efficiency and faster inference in large-scale applications
\end{itemize}

This componentization approach could address challenges in deploying large models on resource-constrained devices or in latency-sensitive applications.
